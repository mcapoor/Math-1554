% !TEX TS-program = pdflatex
% !TEX encoding = UTF-8 Unicode

% This is a simple template for a LaTeX document using the "article" class.
% See "book", "report", "letter" for other types of document.

\documentclass[12pt]{article} % use larger type; default would be 10pt

\usepackage[utf8]{inputenc} % set input encoding (not needed with XeLaTeX)

%%% Examples of Article customizations
% These packages are optional, depending whether you want the features they provide.
% See the LaTeX Companion or other references for full information.

%%% PAGE DIMENSIONS
\usepackage{geometry} % to change the page dimensions
\geometry{letterpaper} % or letterpaper (US) or a5paper or....
% \geometry{margin=2in} % for example, change the margins to 2 inches all round
% \geometry{landscape} % set up the page for landscape
%   read geometry.pdf for detailed page layout information

\usepackage{graphicx} % support the \includegraphics command and options
\usepackage{parskip}
% \usepackage[parfill]{parskip} % Activate to begin paragraphs with an empty line rather than an indent

%%% PACKAGES
\usepackage{booktabs} % for much better looking tables
\usepackage{array} % for better arrays (eg matrices) in maths
\usepackage{paralist} % very flexible & customisable lists (eg. enumerate/itemize, etc.)
\usepackage{verbatim} % adds environment for commenting out blocks of text & for better verbatim
\usepackage{subfig} % make it possible to include more than one captioned figure/table in a single float

% These packages are all incorporated in the memoir class to one degree or another...

%%% HEADERS & FOOTERS
\usepackage{fancyhdr} % This should be set AFTER setting up the page geometry
\pagestyle{fancy} % options: empty , plain , fancy
\renewcommand{\headrulewidth}{0pt} % customise the layout...
\lhead{}\chead{}\rhead{}
\lfoot{}\cfoot{\thepage}\rfoot{}

%%% SECTION TITLE APPEARANCE
\usepackage{sectsty}
\allsectionsfont{\sffamily\mdseries\upshape} % (See the fntguide.pdf for font help)
% (This matches ConTeXt defaults)

%%% ToC (table of contents) APPEARANCE
\usepackage[nottoc,notlof,notlot]{tocbibind} % Put the bibliography in the ToC
\usepackage[titles,subfigure]{tocloft} % Alter the style of the Table of Contents
\renewcommand{\cftsecfont}{\rmfamily\mdseries\upshape}
\renewcommand{\cftsecpagefont}{\rmfamily\mdseries\upshape} % No bold!

\usepackage{amsmath}
\usepackage{amssymb}
%%% END Article customizations

\newcommand{\R}{\mathbb{R}}
\newcommand{\mateq}{$A \vec{x} = \vec{b}$}
%%% The "real" document content comes below...
\pagenumbering{arabic}

\title{Math 1554}
\author{Milan Capoor}
\date{Fall 2021} % Activate to display a given date or no date (if empty),
         % otherwise the current date is printed 

\begin{document}
\maketitle

\section{Module 1: Linear Equations}
\subsection{TOPIC 1: Systems of Linear Equations}
\emph{Linear equation:} an equation in the form $a_1 x_1 + a_2 x_2 + ... + a_n x_n = b$ where $a_n$ and $b$ are real or complex coefficients of the variables $x_n$\\

\emph{Linear system (system of linear equations):} a collection of linear equations involving the same variables\\

\emph{Solution:} a list of values ($s_1, s_2, ..., s_n$) for which substitution into the variables 
($x_1, x_2, ... , x_n$) makes the equations true\\
\indent - Found where the equations’ lines intersect \\

\emph{Solution set:} the collection of all solutions to a linear system\\

\emph{Equivalent:} characteristic of two linear systems if they have the same solution sets\\

\emph{Consistent system:} a linear system which has infinitely many solutions or one solution\\
\indent - Happens when the lines intersect at a single point or when the lines coincide\\

\emph{Inconsistent system:} system with no solutions\\
\indent - When the lines are parallel\\



\emph{Matrix:} the essential information of a linear system can be recorded compactly in a rectangular array\\
Given the system
\begin{align*}
x_1 - 2x_2 + x_3 &= 0\\
\;	2x_2 - 8x_3 &= 8\\
5x_1\; 	- 5x_3 &= 10\\
\end{align*}
The \emph{coefficient matrix}
\begin{center}
	$\begin{bmatrix}
		1 & -2 & 1\\ 
		0 & 2 & -8 \\
		5 & 0 & -5\\
	\end{bmatrix}$
\end{center}
encodes the essential information of the system while the \emph{augmented matrix} is \\
\begin{center}
	$\begin{bmatrix}
		1 & -2 & 1 & 0\\ 
		0 & 2 & -8 & 8\\
		5 & 0 & -5 & 10\\
	 \end{bmatrix}$
\end{center}

An $m \times n$ matrix is a rectangular array of numbers with $m$ rows and $n$ columns.

\subsubsection{Solving a linear system}
The basic strategy is to \emph{replace one system with an equivalent system that is easier to solve}. Three basic operations are used to simplify a linear system: 
\begin{enumerate}
\item (Replacement) Replace one equation by the sum of itself and a multiple of another equation 
\item (Interchange) Swap two equations
\item (Scaling) Multiply all the terms in an equation by a nonzero constant
\end{enumerate}

\subsubsection{Example 1: Solve the system given above}
\begin{align*}
	x_1 - 2x_2 + x_3 &= 0\\
	\;	2x_2 - 8x_3 &= 8\\
	5x_1\; 	- 5x_3 &= 10\\
\end{align*}

\indent Row Reduction Procedure:
\begin{enumerate}
	\item Construct the augmented matrix
		\begin{center}
			$\begin{bmatrix}
				1 & -2 & 1 & 0\\ 
				0 & 2 & -8 & 8\\
				5 & 0 & -5 & 10\\
			 \end{bmatrix}$
		\end{center}
	\item $R_3 - 5R_1$
		\begin{center}
			$\begin{bmatrix}
				1 & -2 & 1 & 0\\ 
				0 & 2 & -8 & 8\\
				0 & 10 & -10 & 10\\
			 \end{bmatrix}$
		\end{center}
	\item $\frac{1}{2} R_2$
		\begin{center}
			$\begin{bmatrix}
				1 & -2 & 1 & 0\\ 
				0 & 1 & -4 & 4\\
				5 & 0 & -5 & 10\\
			 \end{bmatrix}$
		\end{center}
	\item $R_3 - 10R_2$
		\begin{center}
			$\begin{bmatrix}
				1 & -2 & 1 & 0\\ 
				0 & 1 & -4 & 4\\
				0 & 0 & 30 & -30\\
			 \end{bmatrix}$
		\end{center}
	\item $\frac{1}{30}R_3$
		\begin{center}
			$\begin{bmatrix}
				1 & -2 & 1 & 0\\ 
				0 & 1 & -4 & 4\\
				0 & 0 & 1 & -1\\
			 \end{bmatrix}$
		\end{center}
	\item $R_1 - R_3 and R_2 + 4R_3$
		\begin{center}
			$\begin{bmatrix}
				1 & -2 & 0 & 1\\ 
				0 & 1 & 0 & 0\\
				0 & 0 & 1 & -1\\
			 \end{bmatrix}$
		\end{center}
	\leavevmode \\
	\item $R_1 + 2R_2$
		\begin{center}
			$\begin{bmatrix}
				1 & 0 & 0 & 1\\ 
				0 & 1 & 0 & 0\\
				0 & 0 & 1 & -1\\
			 \end{bmatrix}$
		\end{center}
	\item Verify the solution $(1, 0, -1)$ by substituting it for $(x_1, x_2, x_3)$ in the original system
		\begin{align*}
			(1) - 2(0) + (-1) &= 0\\
			2(0) - 8(-1) &= 8\\
			5(1) - 5(-1) &= 10\\
		\end{align*}
\end{enumerate}
The equations agree so $(1, 0, -1)$ is indeed a solution to the system. $\blacksquare$\\

These row operations can be applied to any matrix \textemdash not just those that arise as the augmented matrices of a linear system.\\

\emph{Row equivalent:} quality of two matrices if there exists a sequence of elementary row operations that transforms one matrix into the other\\

\textbf{All row operations are reversible. Hence, \emph{If the augmented matrices of two linear systems are row equivalent, then the two systems have the same solution set.}}

\subsubsection{Existence and Uniqueness}
A large focus of the course and of the analysis of linear systems generally depends on asking two questions:
\begin{itemize}
	\item Is the system consistent?
	\item If a solution exists, is it unique?
\end{itemize}
These questions can often be answered from the triangular form of the matrix (in Example 1 this was $\begin{bmatrix}1 & -2 & 1 & 0\\ 0 & 1 & -4 & 4\\0 & 0 & 1 & -1\\ \end{bmatrix}$) by understanding that $x_3 = -1$ and solving the other equations from there. \\
If this method creates contradictions, however, such as in the system $\begin{bmatrix}a & b & c & d\\ e & f & g & h\\0 & 0 & 0 & 1\\ \end{bmatrix}$, then the originial system is inconsistent. \\

%----------------------------------------------------------------------------------------------------------------------------------------%
\subsection{TOPIC 2: Row Reduction and Echelon Forms}
\emph{Nonzero:} any row or column which contains at least one nonzero entry\\

\emph{Leading entry:} the leftmost nonzero entry of a row\\

\emph{Row echelon form:} a matrix is in row echelon form if it has the following three properties:
\begin{enumerate}
	\item All nonzero rows are above any rows of all zeros
	\item Each leading entry of a row is in a column to the right of the leading entry of the row above it
	\item All entries in a column below a leading entry are zeroes
\end{enumerate}
\emph{Reduced echelon form:} a matrix in row echelon form which satisfies the additional criteria:
\begin{enumerate}
\addtocounter{enumi}{3}
\item The leading entry in each nonzero row is 1
\item Each leading 1 is the only nonzero entry in its column
\end{enumerate}
These are the "triangular" matrices of section 1.1.\\

\textbf{\emph{Each matrix is row equivalent to one and only one reduced echelon matrix.}}\\

If a matrix $A$ is row equivalent to an echelon matrix $U$, we call $U$ an \emph{(reduced) echelon form (REF/RREF) of $A$}\\

\subsubsection{Pivot Positions}
\emph{Pivot position:} a location in a matrix $A$ that corresponds to a leading 1 in the reduced echelon form of $A$.\\

\emph{Pivot column:} a column of $A$ that contains a pivot position\\
\subsubsection{Example 2: Row reduce matrix A below to echelon form and locate its pivot columns.}
\begin{center}
	$A = \begin{bmatrix}
			0 & -3 & -6 & 4 & 9\\
			-1 & -2 & -1 & 3 & 1\\
			-2 & -3 & 0 & 3 & -1\\
			1 & 4 & 5 & -9 & -7\\
		\end{bmatrix}$
\end{center}
Solution:
\begin{enumerate}
\item Interchange $R_1$ and $R_4$
	\begin{center}
		$\begin{bmatrix}
			1 & 4 & 5 & -9 & -7\\
			-1 & -2 & -1 & 3 & 1\\
			-2 & -3 & 0 & 3 & -1\\
			0 & -3 & -6 & 4 & 9\\
		\end{bmatrix}$
	\end{center}
\item $R_2 + R_1$ and $R_3 + 2R_1$
	\begin{center}
		$\begin{bmatrix}
			1 & 4 & 5 & -9 & -7\\
			0 & 2 & 4 & -6 & -6\\
			0 & 5 & 10 & -15 & -15\\
			0 & -3 & -6 & 4 & 9\\
		\end{bmatrix}$
	\end{center}
\item $R_3 -\frac{5}{2}R_2$ and $R_4 + \frac{3}{2}R_2$
	\begin{center}
		$\begin{bmatrix}
			1 & 4 & 5 & -9 & -7\\
			0 & 2 & 4 & -6 & -6\\
			0 & 0 & 0 & 0 & 0\\
			0 & 0 & 0 & -5 & 0\\
		\end{bmatrix}$
	\end{center}
\leavevmode \\
\item Interchange $R_3$ and $R_4$
	\begin{center}
		$\begin{bmatrix}
			1 & 4 & 5 & -9 & -7\\
			0 & 2 & 4 & -6 & -6\\
			0 & 0 & 0 & -5 & 0\\
			0 & 0 & 0 & 0 & 0\\
		\end{bmatrix}$
	\end{center}	
\end{enumerate}
Matrix $A$ is thus in echelon form and so columns 1, 2, and 4 are pivot columns.\\

\emph{Pivot:} a nonzero number in a pivot position which is used during row reduction to create zeros\\

\subsubsection{The Row Reduction Algorithm}
\begin{enumerate}
	\item Begin with the leftmost nonzero column. This is a pivot column. The pivot position is at the top.
	\item Select a nonzero entry in the pivot column as a pivot. If necessary, interchange rows to move this entry into the pivot position. 
	\item Use row replacement operations to create zeros in all positions below the pivot.
	\item Cover the row containing the pivot position and all rows above it. Apply steps 1-3 to the remaining submatrix, iterating until there are no more nonzero rows to modify. At this time we have the row echelon form.
	\item This step is used to find the reduced row echelon form. Beginning with the righmost pivot and working upward and to the left, create zeros above each pivot. If a pivot is not 1, make it 1 via scaling operation.
\end{enumerate}
\leavevmode \\
\subsubsection{Solutions of Linear Systems}
The row reduction algorithm leads directly to the solution set of a system when the algorithm is applied to the augmented matrix of the system.\\

\emph{Basic variables:} the variables of a linear system corresponding to pivot columns in the reduced echelon matrix \\

\emph{Free variables:} the other variables in the system\\

Whenever a system is consistent, the solution set can be described explicitly by solving the reduced system of equations for the basic variables in terms of the free variables.\\
Each different choice of a free variable determines a (different) solution of the system and every solution of the system is determined by a choice of the free variables.

\subsubsection{Example 4: Find the general solution of the linear system whose augmented matrix has been reduced to}
\begin{center}
	$\begin{bmatrix}
		1 & 6 & 2 & -5 & -2 & -4\\
		0 & 0 & 2 & -8 & -1 & 3\\
		0 & 0 & 0 & 0 & 1 & 7
	\end{bmatrix}$
\end{center}
Moving from echelon form to reduced echelon form we have:\\

$\begin{bmatrix}
	1 & 6 & 2 & -5 & -2 & -4\\
	0 & 0 & 2 & -8 & -1 & 3\\
	0 & 0 & 0 & 0 & 1 & 7
\end{bmatrix}$
\textasciitilde 
$\begin{bmatrix}
	1 & 6 & 2 & -5 & 0 & 10\\
	0 & 0 & 2 & -8 & 0 & 10\\
	0 & 0 & 0 & 0 & 1 & 7
\end{bmatrix}$
\textasciitilde 
$\begin{bmatrix}
	1 & 6 & 2 & -5 & 0 & 10\\
	0 & 0 & 1 & -4 & 0 & 5\\
	0 & 0 & 0 & 0 & 1 & 7
\end{bmatrix}$
\textasciitilde
$\begin{bmatrix}
	1 & 6 & 0 & 3 & 0 & 0\\
	0 & 0 & 1 & -4 & 0 & 5\\
	0 & 0 & 0 & 0 & 1 & 7\\
\end{bmatrix}$



Which is equivalent to 
\begin{align*}
	x_1 + 6x_2 + 3x_4 &= 0\\
	x_3 - 4x_4 &= 5 \\
	x_5 &= 7\\
\end{align*}
Because the pivot columns are 1, 3, and 5, the basic variables are $x_1, x_3, and x_5$, leaving $x_2$ and $x_4$ as free variables. Thus
\begin{center}
	$\begin{cases}
		x_1 = -6x_2 - 3x_4\\
		x_2 \text{ is free}\\
		x_3 = 5 + 4x_4\\
		x_4 \text{ is free}\\
		x_5 = 7\\
	\end{cases}$
\end{center}
\subsubsection{Existence and Uniqueness Questions (Redux)}
When a system is in echelon form and contains no equation of the form $0 = b (b \neq 0)$, \emph{every} nonzero equation contains a basic variable with a nonzero coefficient. Either the basic variables are completely determined (no free variables), signifying a unique solution, or the basic variables can be expressed in terms of one or more free variables, signifying infinitely many solutions. Thus:\\
\textbf{A linear system is consistent iff the rightmost column of the augmented matrix is \emph{not} a pivot column (i.e. there are no rows of the form $\begin{bmatrix} 0 & ... & 0 & b\end{bmatrix}$).}


%----------------------------------------------------------------------------------------------------------------------------------------%
\subsection{TOPIC 3: Vector Equations}
\emph{Column vector:} a matrix with only one column
$$\mathbf{w}= \begin{bmatrix}w_1 \\ w_2 \end{bmatrix}$$
The set of all vectors with two entries is denoted $\mathbb{R}^2$ ("r-two"), where $\mathbb{R}$ refers to the real number entries of the vector and the exponent 2 indicates that each vector contains two entries.\\
Vectors in $\mathbb{R}$ are ordered pairs which are equal iff their corresponding entries are equal.\\

Given two vectors \textbf{u} and \textbf{v} in $\R^2$, their sum is 
\begin{center}\textbf{u} + \textbf{v} = $\begin{bmatrix}u_1 + v_1 \\ u_2 + v_2\end{bmatrix}$\end{center}
while the scalar multiple of a vector is 
\begin{center}c\textbf{u} = c$\begin{bmatrix}u_1 \\ u_2\end{bmatrix} =\begin{bmatrix}c\cdot u_1 \\ c\cdot u_2\end{bmatrix}  $\end{center}

\subsubsection{Geometric Descriptions of $\R^2$}
Because each point in the cartesian plane is determined by an ordered pair of numbers, we can identify a geometric point with the column vector $\begin{bmatrix}a \\ b\end{bmatrix}$. So we may regard $\R^2$ as the set of all points in the plane.\\

\emph{Parellelogram rule for vector addition:} If \textbf{u} and \textbf{v} in $\R^2$ are represented as points in the plane, then $\textbf{u} + \textbf{v}$ corresponds to the fourth vertex of the parellolgram whose other vertices are $\textbf{u}$, $\textbf{0}$, and $\textbf{v}$.

\subsubsection{Algebraic Properties of $\R^n$}
For all $\bf{u}, \bf{v}, \bf{w}$ in $\R^n$ and all scalars $c$ and $d$:
\begin{enumerate}
	\item $\bf{u} + \bf{v} = \bf{v} + \bf{u}$
	\item $(\bf{u + v) + w = u + (v + w)}$
	\item $\bf{u + 0 = 0 + u = u}$
	\item $\bf{u + (-u) = -u + u = 0}$
	\item $c(\bf{u + v}$) = c$\bf{u}$ + c$\bf{v}$
	\item $(c + d)\bf{u} = c\bf{u} + d\bf{u}$
	\item $c(d\bf{u}) = (cd)\bf{u}$
	\item $1\bf{u = u}$
\end{enumerate}
\leavevmode \\
\emph{Linear combination:} a composite formed of vectors in $\R^n$ and corresponding scalar weights ($\bf{y}$$=c_1$$\bf{v_1+ ... + }$$c_p$$\bf{v_p}$)

\subsubsection{Example 5: Solve $x_1\bf{a_1}$$ + x_2 $$\bf{a_2} = \bf{b} \text{ for } \bf{a_1} = \begin{bmatrix}1\\-2\\-5\end{bmatrix}, \bf{a_2} = \begin{bmatrix}2\\5\\6\end{bmatrix}\text{, and  }\bf{b} = \begin{bmatrix}7\\4\\-3\end{bmatrix}.$}
We shall begin by constructing an augmented matrix and finding its reduced row echelon form:
\begin{center}
	$\begin{bmatrix}
		1 & 2 & 7\\
		-2 & 5 & 4\\
		-5 & 6 & -3
	\end{bmatrix}$
	\textasciitilde
	$\begin{bmatrix}
		1 & 2 & 7\\
		0 & 9 & 18\\
		0 & 16 & 32
	\end{bmatrix}$
	\textasciitilde
	$\begin{bmatrix}
		1 & 2 & 7\\
		0 & 1 & 2\\
		0 & 1 & 2
	\end{bmatrix}$
	\textasciitilde
	$\begin{bmatrix}
		1 & 2 & 7\\
		0 & 1 & 2\\
		0 & 0 & 0\\
	\end{bmatrix}$
\end{center}
So $x_2 = 2$ and $x_1 = 7 - 2x_2 = 3$.\\

Checking: $3\begin{bmatrix}1\\-2\\-5\end{bmatrix} + 2\begin{bmatrix}2\\5\\6\end{bmatrix} = \begin{bmatrix}3 + 4\\-6 + 10\\-15 + 12\end{bmatrix} = \begin{bmatrix}7\\4\\-3\end{bmatrix} \blacksquare.$\\





\emph{A vector $\bf{b}$ can be generated by a linear combination of $\bf{a_1, ..., a_n}$ iff there exists a solution to the linear system corresponding to the augmented matrix $\begin{bmatrix}\bf{a_1} & \bf{a_2} & ... & \bf{a_n} & \bf{b}\end{bmatrix}$.}\\
One of the key ideas in linear algebra is to study the set of all vectors that can be written as a linear combination of a fixed set $\{ \bf{v_1, ..., v_p} \}$ of vectors.\\

\emph{The subset of $\R^n$ spanned by $\bf{v_1}, ..., \bf{v_p}:$} (written Span$\{\bf{v_1, ..., v_p}\}$) is the collection of all vectors that can be written in the form $c_1$$\bf{v_1 }$$ + c_2$$\bf{v_2 }$$ + ... + c_p\bf{v_p}$ with $c_1, ..., c_p$ scalars.\\

Asking whether a vector is in Span$\{\bf{v_1, ..., v_p}\}$ amounts to asking whether the linear system with associated augmented matric has a solution.\\

Note: Span$\{\bf{v_1, ..., v_p}\}$ contains every scalar multiple of $\bf{v_1}$ including $\bf{0}$
\subsubsection{Geometric description of Span$\{\bf{v}\}$}
For $\bf{v} \in \R^3$, Span$\{\bf{v}\}$ is the set of points on the line in $\R^3$ through \textbf{v} and \textbf{0}.\\
For $\textbf{u, v} \in \R^3$, Span$\{\textbf{u, v}\}$ is the plane in $\R^3$ that contains \textbf{u}, \textbf{v}, and \textbf{0}. \\

\subsubsection{Example 6: Span$\{\mathbf{a_1, a_2}\}$ is a plane through the origin in 
$\R^3$. Is $\mathbf{b}$ in that plane?}
\begin{center}
	($\mathbf{a_1} = \begin{bmatrix}1\\-2\\3\end{bmatrix}, 
	\mathbf{a_2} = \begin{bmatrix}5\\-13\\-3\end{bmatrix}, 
	\text{and }\mathbf{b} = \begin{bmatrix}-3\\8\\1\end{bmatrix}$)
\end{center}

This question amounts to asking if the equation $x_1\mathbf{a_1} + x_2\mathbf{a_2} = \mathbf{b}$ has a solution. From row reduction:
\begin{center}
	$\begin{bmatrix}
		1 & 5 & -3\\
		-2 & -13 & 8\\
		3 & -3 & 1\\
	\end{bmatrix}
	\sim
	\begin{bmatrix}
		1 & 5 & -3\\
		0 & -3 & 2\\
		0 & -18 & 10\\
	\end{bmatrix}
	\sim
	\begin{bmatrix}
		1 & 5 & -3\\
		0 & -3 & 2\\
		0 & 0 & -2\\
	\end{bmatrix}$
\end{center}
This suggests that $0 = -2$ but as this is not possible, the matrix is inconsistent and so the vector equation has no solution. Thus, $\mathbf{b}$ is not in Span\{\textbf{$a_1, a_2$}\}. $\blacksquare$



%----------------------------------------------------------------------------------------------------------------------------------------%
\subsection{TOPIC 4: The Matrix Equation}
\begin{itemize}
\item $\in$: \indent belongs to \\
\item $\R^n$: \indent the set of vectors with \emph{n} real-valued elements\\
\item $\R^{m\times n}$: \indent the set of real-valued matrices with \emph{m} as rows and \emph{n} as columns\\
\end{itemize}
Example:
The notation $\vec{x} \in \R^5 m$ measn that $\vec{x}$ is a vector with five real-valued elements.\\

\subsubsection{Matrix-Vector Product as a Linear Combination}
If $A \in \R^{m\times n}$ has columns $\vec{a_1}, ..., \vec{a_n}$, then the matrix vector product $A\vec{x}$ is a linear combination of the columns of \emph{A}. 
$$A\vec{x} = \sum_{n=1}^n x_n \vec{a_n}$$
Note that $A\vec{x}$ is in the span of the columns of A.\\

This means that the solution sets for 
$$A\vec{x} = \vec{b}$$
is the same as
$$x_1\vec{a_1} + ... + x_n\vec{a_n} = \vec{b}$$
which is again equivalent to the set of linear equations with the augmented matrix
$$\begin{bmatrix}
	\vec{a_1} & \vec{a_2} & ... & \vec{a_n} & \vec{b_n}
\end{bmatrix}$$

\subsubsection{Example:} 
Suppose that $A = \begin{bmatrix}1 & 0\\ 0 & -3\end{bmatrix}$ and $\vec{x} = \binom{2}{3}$\\
\begin{enumerate}
\item The following product can be written as a linear combination of vectors:
$$A\vec{x} = 2 \binom{1}{0} + 3 \binom{0}{-3} = \binom{2}{-9}$$

\item Is $\vec{b} = \binom{2}{9}$ in the span of the columns of A?\\
If $\vec{b} \in Span\{A\}$, then $\vec{b} = c_1 \binom{1}{0} + c_2 \binom{0}{-3}$. This is true for $\vec{c} = \binom{2}{-3}$ so $\vec{b} \in Span\{A_{col}\}$
\end{enumerate}

\textbf{The equation \mateq  has a solution iff $\vec{b}$ is a linear combination of the columns of $A$}

\subsubsection{Example: For what vectors
$\vec{b} = \begin{bmatrix}b_1\\b_2\\b_3\\\end{bmatrix}$ does the equation have a solution?}
$$ \begin{bmatrix}
	1&3&4\\
	2&8&4\\
	0&1&-2\\
\end{bmatrix} \vec{x} = \vec{b}$$

Solution: 
$$\begin{bmatrix}
	1 & 3 & 4 & b_1\\
	2 & 8 & 4 & b_2\\
	0 & 1& -2 & b_3\\
\end{bmatrix} \sim 
\begin{bmatrix}
	1 & 3 & 4 & b_1\\
	0 & 2 & -4 & b_2 - 2b_1\\
	0 & 1& -2 & b_3\\
\end{bmatrix} \sim 
\begin{bmatrix}
	1 & 3 & 4 & b_1\\
	0 & 2 & -4 & b_2 - 2b_1\\
	0 & 0 & 0 & b_3 - \frac{1}{2}b_2 + b_1\\
\end{bmatrix}$$
So $$ \vec{b} = \begin{bmatrix} -\frac{1}{2}b_2 + b_3\\ b_2 \\ b_3\end{bmatrix}$$

Essential concept:
If $A$ is an $m\times n$ matrix, the following statements are logically equivalent -- for a particular $A$, \emph{all} are true or \emph{all} are false:
\begin{itemize}
	\item For each \textbf{b} in $\R^m$, the equation $A\mathbf{x} = \mathbf{b}$ has a solution.
	\item Each \textbf{b} in $\R^m$ is a linear combination of the columns of $A$
	\item The columns of $A$ span $\R^m$
	\item $A$ has a pivot position in every row
\end{itemize}

\subsubsection{Summary: Ways of representing Linear Systems}
\begin{enumerate}
	\item A list of equations
	\item An augmented matrix
	\item A vector equation
	\item A matrix equation
\end{enumerate}

\subsubsection{Matrix-vector products}
If $A$ is an $m\times n$ matrix, \textbf{u} and \textbf{b} are vectors in $\R^n$, and $c$ is a scalar, then:
\begin{itemize}
	\item $A(\mathbf{u} + \mathbf{v}) = A\mathbf{u} + A\mathbf{v}$
	\item $A(c\mathbf{u}) = c(A\mathbf{u})$
\end{itemize}

Moreover, the product $A\mathbf{x}$ can be easily calculated by taking advantage of the nature of a Matrix-vector product:
$$\begin{bmatrix}
	a_1 & a_2 & a_3\\
	a_4 & a_5 & a_6\\
	a_7 & a_8 & a_9\\
\end{bmatrix}\begin{bmatrix}
	x_1\\
	x_2\\
	x_3\\
\end{bmatrix} = \begin{bmatrix}
	a_1 x_1 + a_2 x_2 + a_3 x_3\\
	a_4 x_1 + a_5 x_2 + a_6 x_3\\
	a_7 x_1 + a_8 x_2 + a_9 x_3\\
\end{bmatrix}$$

\emph{The identity matrix:} denoted $\mathbf{I}$, this $n\times n$ matrix contains 1's on the diagonal and 0's elsewhere, creating the universal property that $\mathbf{I_n x} = \mathbf{x} \text{  for every  } \mathbf{x} \in \R^n$

%----------------------------------------------------------------------------------------------------------------------------------------%
\subsection{TOPIC 5: Solution sets of Linear Systems}
\emph{Homogenous:} characteristic of linear systems of the form \mateq, $\vec{b} = \vec{0}$\\

\emph{Inhomogeneous:} systems of the form \mateq, $\vec{b} \neq \vec{0}$\\

Because homogenous systems always have trivial solutions, the interesting question comes in asking whether they have non-trivial solutions.


\begin{center}
	$A \vec{x} = 0$ has a nontrivial solution $\iff$ there is a free variable $\iff A$ has a column with no pivot
\end{center}


\subsubsection{Example: Identify the free variables and the solution set for}
\begin{align*}
	x_1 + 3x_2 + x_3 &= 0\\
	2x_1 - x_2 - 5x_3 &= 0\\
	x_1 - 2x_3 &= 0
\end{align*}

$$\begin{bmatrix}
	1 & 3 & 1 & 0\\
	2 & -1 & -5 & 0\\
	1 & 0 & -2 & 0
\end{bmatrix} \sim
\begin{bmatrix}
	1 & 3 & 1 & 0\\
	0 & -7 & -7 & 0\\
	0 & -3 & -3 & 0
\end{bmatrix} \sim
\begin{bmatrix}
	1 & 3 & 1 & 0\\
	0 & 1 & 1 & 0\\
	0 & 1 & 1 & 0
\end{bmatrix} \sim
\begin{bmatrix}
	1 & 0 & -2 & 0\\
	0 & 1 & 1 & 0\\
	0 & 0 & 0 & 0\\
\end{bmatrix}$$

Row 3 has no pivot ($x_3$ is free) so there is a non-trivial solution. 
$$
\begin{cases}
	x_1 = &2x_3\\
	x_2 = &-x_3\\
	x_3 &\text{is free}
\end{cases} \implies x_3 \begin{bmatrix} 2 \\ -1\\ 1\end{bmatrix}$$
\pagebreak
\subsubsection{Parametric vector forms}
\emph{Parametric vector form:} a more convenient way of expressing the solutions of a linear system, taking advantage of the geometric interpretation of a linear system. In general, for free variables $x_k, ... x_n$ of $A\vec{x} = 0$, the solutions can all be written as $\vec{x} = \sum_{n=k}^n x_n\vec{v_n}$

$$\begin{bmatrix}
	1 & 3 & 1 & 0\\
	2 & -1 & -5 & 0\\
	1 & 0 & -2 & 0
\end{bmatrix} \sim
\begin{bmatrix}
	1 & 0 & -2 & 0\\
	0 & 1 & 1 & 0\\
	0 & 0 & 0 & 0\\
\end{bmatrix}$$
\begin{align*}
	x_1 - 2x_3 &= 0\\
	x_2 + x_3 &= 0\\
	0 &= 0
\end{align*}
So $x_3$ is a free variable. Thus, rearranging we have
$$\vec{x} = 
	\begin{bmatrix}
		x_1 \\ x_2 \\ x_3
	\end{bmatrix} 
	= 
	\begin{bmatrix}
		2x_3 \\ - x_3 \\ x_3
	\end{bmatrix}
	= 
	x_3\begin{bmatrix}2\\-1\\1\end{bmatrix}
$$


%----------------------------------------------------------------------------------------------------------------------------------------%

\subsection{TOPIC 6: Linear Independence}
A set of vectors {$\vec{v_1}, ..., \vec{v_k}$} $\in \R^n$ are linearly independent if $\sum_{n=1}^k = c_k \vec{v_k} = 0$ has only the trivial solution ($\vec{c} = \vec{0}$). It is linearly dependent otherwise. 

Establishing linear independence is thus equivalent to asking whether the equation $V\vec{c} = 0$ ($V \in \R^k :=$ the matrix corresponding to the linear combinations of the vectors) is only true for $\vec{c} = \vec{0}$

Example: For what values of $h$ is the set of vectors linearly independent? 
$$
	\begin{bmatrix}
		1\\1\\h
	\end{bmatrix}, 
	\begin{bmatrix}
		1\\h\\1
	\end{bmatrix}, 
	\begin{bmatrix}
		h\\1\\1
	\end{bmatrix}
$$
$$
\begin{bmatrix}
	1 & 1 & h & 0\\
	1 & h & 1 & 0\\
	h & 1 & 1 & 0\\
\end{bmatrix} \sim 
\begin{bmatrix}
	1 & 1 & h & 0\\
	0 & h - 1 & 1 - h & 0\\
	0 & 1 - h & 1 - h^2 & 0\\
\end{bmatrix} \sim 
\begin{bmatrix}
	1 & 1 & h & 0\\
	0 & h - 1 & 1 - h & 0\\
	0 & 0 & 2 - h - h^2 & 0\\
\end{bmatrix}
$$

If $2 - h - h^2 = 0$ then we have a free variable and the vectors will be linearly dependent (because there will be more solutions than the trivial case).

Factoring, we get 
$$0 = -(h + 2)(h - 1)$$
so for the vectors to be independent, 
$$h \neq \{-2, 1\}$$

\subsubsection{Linear Independence Theorems}
\begin{enumerate}
	\item More Vectors Than Elements
		\indent For vectors $\vec{v}_1, ..., \vec{v}_k \in \R^n$, if $k > n$, then $\{\vec{v}_1, ..., \vec{v}_k\}$ is linearly dependent (because not every column of the matrix $A = (\vec{v}_1, ..., \vec{v}_k)$ would be pivotal). 
	\item Set Contains Zero Vector
		\indent If any one or more of $\vec{v}_1, ..., \vec{v}_k$ is $\vec{0}$, then $\{\vec{v}_1, ..., \vec{v}_k\}$ is linearly dependent (again because there would be non-pivotal columns of the corresponding matrix).


\end{enumerate}
\end{document}
